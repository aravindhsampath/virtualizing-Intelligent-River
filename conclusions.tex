\chapter{Conclusion}
%This chapter summarizes the problems addressed by the thesis and the corresponding contributions.
%\section{Summary}

Wireless sensor networks are used to gather meaningful data and enable important applications. They are important tools for monitoring the physical world. However, to be effective, they must satisfy some basic conditions. Specifically, they must be \textit{maintenance free}, \textit{inexpensive}, \textit{reliable}, and \textit{scalable}. The devices used in these networks are deployed in numbers ranging from the hundreds to thousands, and they often rely on battery power. Replacing the batteries in such large networks is expensive, both in terms of battery cost and personnel time. The cost of an individual device also plays an important role in the cost of a sensor network. Further, if these networks are deployed in safety-critical contexts, the risk of missing data or collecting incorrect data must be minimized. Finally, a sensor network must be able to accommodate any number of devices and gracefully handle device failure.

To address these issues, our solution uses a multi-faceted approach. First, to minimize the cost of maintaining a network, we design a maintenance free mote. To accomplish this, our sensing platform includes an energy harvesting circuit which harvests solar energy and stores it in a Li-Ion battery. This stored energy is used whenever there is insufficient solar power. Further, we use custom software to put the controller in deep sleep state whenever possible to save energy. Second, we focus on a low cost mote designed to sense the physical and chemical parameters in a micro-environment. We use components which are low in cost, but sufficient for a simplified mote design. Instead of using common, expensive radios, we use the RFM12, which offers high data-rates, variable transmission power, multi-channel operation, and long range, at a low cost. The RFM12 does not require any external components to transmit or receive data, which further simplifies our design. Third, we develop a network design to transmit and receive data reliably. To achieve this, we design a protocol based on a lightweight TDMA strategy with packet acknowledgments to ensure that every packet reaches the base station. Fourth, we focus on a scalable solution which accommodates the addition of new devices and handles mote failures. In addition, we also develop a lightweight time synchronization protocol and a route formation protocol. We then evaluate our solution on the basis of reliability, scalability, and network longevity.

%This thesis describes four contributions. First, we present a maintenance-free network. We accomplish this by harvesting solar energy. Solar energy harvesting eliminates the need to replace the batteries of motes. Second, we present an inexpensive network. We accomplish this by designing motes which use basic electronic components, such as resistors, capacitors, and diodes, instead of using more costly integrated circuits. Third, we present a reliable network. We accomplish this by developing a lightweight, TDMA based protocol that uses acknowledgments to ensure reliable data transmission and reception. Fourth, we present a scalable network that handles a dynamic number of motes. We believe that this solution will help users construct maintenance free, inexpensive, reliable, and scalable networks.


%This thesis describes four major contributions. First, we describe a maintenance free network. We accomplish this by harvesting solar energy and by efficiently using the available energy. Second, we describe the design of an inexpensive mote. We accomplish this by designing motes which use basic electronic components \textemdash resistors, capacitors, and diodes \textemdash instead of more complex integrated circuits. Third, we describe a reliable network design. We accomplish this by developing a lightweight TDMA protocol with acknowledgments to ensure reliable data transmission and reception. Fourth, we describe a scalable network design that accommodates the dynamic addition and removal of motes without affecting network reliability. In addition to the above contributions, we also develop a lightweight time synchronization protocol and route formation protocol.


This thesis describes a low-cost, maintenance-free sensor networking platform, supported by lightweight, yet reliable and scalable networking software. The hardware/software solution exhibits four key characteristics. First, we present a maintenance free solution. We accomplish this by harvesting solar energy and by efficiently using the available energy. Second, we present an inexpensive solution. We accomplish this by designing a mote platform which uses basic electronic components, such as resistors, capacitors, and diodes instead of using more costly integrated circuits. Third, we present a reliable solution that ensures high yield, even in the presence of  intermittent and permanent device faults. We accomplish this by developing lightweight route formation, time synchronization, and TDMA protocols; the latter includes application-level acknowledgments to ensure reliable data transmission and reception. Finally, we present a scalable solution that accommodates the dynamic addition and removal of motes without affecting network reliability. We believe that this solution will help users construct maintenance free, inexpensive, reliable, and scalable networks.


%This thesis addresses the above problems. First, we address the problem of achieving a maintenance free network. We accomplish this by harvesting solar energy, and by efficiently using the available energy. Solar energy harvesting eliminates the need to change depleted batteries. The evaluation data suggests that our network is maintenance free. Second, we address the problem of achieving a low cost sensor network. We accomplish this by designing motes which uses basic electronic components resistors, capacitors, and diodes, instead of more complex integrated circuits. Third, we address the problem of achieving a reliable network. We accomplish this by developing an acknowledgment based protocol to ensure reliable data transmission and reception. The evaluation data suggests that the network formed using our system is reliable. Fourth, we contribute a scalable network that accommodates dynamic addition of the mote without affecting its reliability. The evaluation data suggests that the network formed using our system is scalable, and accommodates dynamic addition of motes. This thesis also contributes a light weight time synchronization protocol and a network formation protocol.
