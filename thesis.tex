%
% thesis.tex
%
% Master's Thesis/Ph.D. Dissertation Template
% Clemson University
%

%
% The document guidelines say the font can be between 10pt and 12pt.
% Specify whatever you want it to be here.
%
\documentclass[10pt]{ClemsonThesis}

%
% Use any additional packages you might need
%
\usepackage{comment}
\usepackage{listings}
\usepackage{graphicx}
\usepackage{subfig}
\usepackage{algorithmic}
\usepackage{algorithm}
\usepackage{cite}
\usepackage{amsmath}
\usepackage{amssymb}
\usepackage{url}
\usepackage{multirow}

%
% Make the document your own -- fill in these values to reflect the type of
% document you are writing.
%
\title{\textsc{Virtualizing Intelligent River\textsuperscript{\textregistered} : A Comparative Study of Alternative Virtualization Technologies }}
\department{School of Computing}
\documentType{Thesis}
\major{Computer Science}
\degree{Masters of Science}
\graduationMonth{December}
\graduationYear{2013}
\author{Aravindh Sampath Kumar}
\committeeChair{Dr. Jason O. Hallstrom}
\committeeMemberOne{Dr. Amy Apon}
\committeeMemberTwo{Dr. Brian A. Malloy}
%% optional (for Master's) \committeeMemberThree{Dr. John Doe}
%% optional \committeeMemberFour{Dr. Jane Doe}
%% optional \committeeMemberFive{Dr. Mary Doe}
%% optional \committeeMemberSix{Dr. Mark Doe}

%
% PDF Setup -- most of this you do not need to touch
%
\hypersetup{
    colorlinks,
    linkcolor={black},
    citecolor={black},
    filecolor={black},
    urlcolor={black},
    pdftitle={\theTitle},
    pdfauthor={\theAuthor},
    pdfsubject={\theDocumentType},
    pdfkeywords={Clemson University, \theDepartment, \theDocumentType, \theMajor, \theDegree},
    pdfstartpage={1},
}


%
% User-specified command definitions/redefinitions
%
\newcommand{\cplusplus}{{\rm C\raise.5ex\hbox{\small ++}}}
\renewcommand{\ttdefault}{pcr}
\renewcommand\lstlistlistingname{List of Listings}


\begin{document}
%  ============================================================================
    \frontmatter % Begin front matter (pages are numbered with Roman numerals)
%  ============================================================================

    \addtotoc{Title Page}{\maketitle}          % Generate the title page
    \doublespacing                             % Text should be double spaced
    \setcounter{page}{2}                       % Abstract begins on page 2
    \addtotoc{Abstract}{\chapter*{Abstract}



Emerging cloud computing infrastructure models facilitate a modern and efficient way of utilizing computing resources, enabling applications to scale with varying demands over time. The core enabler of the cloud computing paradigm is \textit{virtualization}. The concept is not new; virtualization has garnered significant research attention, fostering a race to achieve the lowest possible virtualization overhead. The evolution has led to three primary virtualization approaches: \textit{full-virtualization}, \textit{para-virtualization}, and \textit{container-based virtualization}, each with a unique set of strengths and weaknesses.  Thus it becomes important to study and evaluate their quantitative and qualitative differences. The operational requirements of the Intelligent River\textsuperscript{\textregistered} middleware system motivated us to compare the choices beyond the standard benchmarks to bring out the unique benefits and limitations of the three virtualization approaches.


This thesis evaluates representative implementations of each approach: (i) full-virtualization - \textit{KVM}, (ii) para-virtualization - \textit{Xen}, and (iii) container-based virtualization - \textit{Linux Containers}. First, this thesis discusses the design principles behind the chosen virtualization solutions. Second, this thesis evaluates the solutions based on the overhead they impose to virtualize system resources. Finally, this thesis assesses the benefits and limitations of each solution based on their operational flexibility, and resource entitlement and isolation facilities.


%this thesis describes the scalable and resilient design of the Intelligent River\textsuperscript{\textregistered} middleware system used as a reference application to evaluate the virtualization platforms. Second, this thesis describes the deployment of the application components on each of the test environments. Third, this thesis assesses the benefits and limitations of each based on their virtualization overhead, resource entitlement and isolation facilities, operational flexibility, scalability, and security.


%This thesis evaluates the effectiveness of a representative each of full-virtualization - KVM, para-virtualization - Xen and container-based virtualization - LXC. First, I describe the design of the Intelligent River middleware system that motivated us to compare the choices. Second, I describe the deployment of the middleware system on each of the test environments. Third, I discuss their quantitative and qualitative differences.

The study presented in this thesis provides an improved understanding of available virtualization technologies. The results will be useful to  system architects in selecting the best virtualization platform for a given application.

% based on their virtualization overhead, resource entitlement and isolation facilities, operational flexibility, scalability and security.


}  % Generate the abstract

    %
    % The dedication page is optional.  Comment out this line if you do not
    % want to include this page.
    %
    %\addtotoc{Dedication}{\chapter*{Dedication}
%\textit{for my parents}

This work is dedicated to my parents, my sister, and many friends whose love, belief, and support enables me to pursue my dreams.  }

    %
    % The acknowledgment page is optional.  Comment out this line if you do
    % not want to include this page.
    %
    %\addtotoc{Acknowledgments}{\chapter*{Acknowledgments}
%I owe my deepest gratitude and respect to my advisor, Dr. Jason O. Hallstrom. If it were not for his ideas, support, guidance and motivation, none of this would have been possible. I would like to thank Dr. Brian Malloy and Dr. Jacob Sorber who served on my committee, and who encouraged and motivated me.

%I thank Yvon Fester and Cullum Smith for their encouragement and suggestions. I am greatly in debt to Ravi and Aravindh for their amazing company. I would also like to thank Yang, Yuheng, Gyan, Neeraj, Sanjay, and Jiannan who always encouraged and supported me. They made graduate school and life much more enjoyable. A special thanks to everyone who wished good for me. Last, but not the least, I thank my parents and brother for all their love, sacrifices, and support.
}

    %\singlespacing                             % Single space the lists
    %\tableofcontents \clearpage                % Generate the Table of Contents
    %\addtotoc{List of Tables}{\listoftables}   % Generate the List of Tables
    %\addtotoc{List of Figures}{\listoffigures} % Generate the List of Figures

    %
    % Include other optional lists.  Computer science, for example, would
    % likely include a 'List of Listings' (and would \usepackage{listings}
    % and \renewcommand\lstlistlistingname{List of Listings}).
    %
%    \addtotoc{List of Listings}{\lstlistoflistings}



%  ===========================================================================
    \mainmatter % Begin main matter (pages are numbered with Arabic numerals)
%  ===========================================================================
    \doublespacing % Text should be double spaced

    %
    % Here we have each chapter in a separate file.  Name these as you choose,
    % and include them in the order you want them to appear.  Be sure to use
    % the \inputfile command.
    %
    %\inputfile{eq.tex}
    \inputfile{introduction.tex}
    \inputfile{relatedWork.tex}
    %\inputfile{hardware.tex}
    \inputfile{virtualization.tex}
    %\inputfile{software.tex}
    %\inputfile{evaluation.tex}
    %\inputfile{results.tex}
    %\inputfile{conclusions.tex}

    %
    % The appendices are optional.  This is the format for two or more.
    % If you do not wish to include an appendix, comment out these lines.
    % If you want just one, see the formatting guidelines.
    %
 %   \begin{appendices}
 %       \begin{subappendices}
 %           \inputfile{appendixA.tex}
 %           \inputfile{appendixB.tex}
 %           \inputfile{appendixC.tex}
 %       \end{subappendices}
 %   \end{appendices}



    \singlespacing                             % Single space the Bibliography

    %
    % The bibliography style.  Set this to whatever matches you discipline.
    % For example, Computer Science would likely use 'plain'.  You might
    % also want to change the name from 'Bibliography' to 'References'.
    %
    % 'plain'   gets you numbered references and citations (e.g., [1] Dyson).
    %
    % 'alpha'   gets you labels formed from an abbreviation of the authors'
    %           names and the year of publication.  If there is more than
    %           one author, it will use the first letter of up to the first
    %           three authors' last names.
    %
    %           Some examples:
    %               [DED01] F.W. Dyson, A.G. Edgar, and D.B. Denny ... 2001
    %               [DE01] F.W. Dyson, A.G. Edgar ... 2001
    %               [Dys01] F.W. Dyson ... 2001
    %
    % 'apalike' gets you labels formed from the authors' names and year of
    %           publication.
    %
    %           Some examples:
    %               [Dyson et al., 2001] F.W. Dyson, A.G. Edgar, and
    %                 D.B. Denny ... 2001
    %               [Dyson and Edgar, 2001] F.W. Dyson, A.G. Edgar ... 2001
    %               [Dyson, 2001] F.W. Dyson ... 2001
    %
    \bibliographystyle{plain}
    \addtotoc{Bibliography}{\bibliography{bibliography}}
\end{document}




