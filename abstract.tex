\chapter*{Abstract}



Emerging cloud computing infrastructure models facilitate a modern and efficient way of utilizing computing resources, enabling applications to scale with varying demands over time. The core enabler of the cloud computing paradigm is \textit{virtualization}. The concept is not new; virtualization has garnered significant research attention, fostering a race to achieve the lowest possible virtualization overhead. The evolution has led to three primary virtualization approaches: \textit{full-virtualization}, \textit{para-virtualization}, and \textit{container-based virtualization}, each with a unique set of strengths and weaknesses.  Thus it becomes important to study and evaluate their quantitative and qualitative differences. The operational requirements of the Intelligent River\textsuperscript{\textregistered} middleware system motivated us to compare the choices beyond the standard benchmarks to bring out the unique benefits and limitations of the three virtualization approaches.


This thesis evaluates representative implementations of each approach: (i) full-virtualization - \textit{KVM}, (ii) para-virtualization - \textit{Xen}, and (iii) container-based virtualization - \textit{Linux Containers}. First, this thesis discusses the design principles behind the chosen virtualization solutions. Second, this thesis evaluates the solutions based on the overhead they impose to virtualize system resources. Finally, this thesis assesses the benefits and limitations of each solution based on their operational flexibility, and resource entitlement and isolation facilities.


%this thesis describes the scalable and resilient design of the Intelligent River\textsuperscript{\textregistered} middleware system used as a reference application to evaluate the virtualization platforms. Second, this thesis describes the deployment of the application components on each of the test environments. Third, this thesis assesses the benefits and limitations of each based on their virtualization overhead, resource entitlement and isolation facilities, operational flexibility, scalability, and security.


%This thesis evaluates the effectiveness of a representative each of full-virtualization - KVM, para-virtualization - Xen and container-based virtualization - LXC. First, I describe the design of the Intelligent River middleware system that motivated us to compare the choices. Second, I describe the deployment of the middleware system on each of the test environments. Third, I discuss their quantitative and qualitative differences.

The study presented in this thesis provides an improved understanding of available virtualization technologies. The results will be useful to  system architects in selecting the best virtualization platform for a given application.

% based on their virtualization overhead, resource entitlement and isolation facilities, operational flexibility, scalability and security.


