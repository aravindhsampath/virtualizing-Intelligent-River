\chapter*{Abstract}



Emerging cloud computing infrastructure models facilitate a modern and efficient way of utilizing virtualized resources, enabling applications to scale with varying demands over time. The core idea behind the cloud computing paradigm is \textit{virtualization}. The concept of virtualization is not new; it has garnered significant research attention, in a race to achieve the lowest overhead compared to bare-metal systems. The evolution has led to three primary virtualization approaches- \textit{full-virtualization}, \textit{para-virtualization}, and \textit{container-based virtualization}, each with a unique set of strengths and weaknesses.  Thus it becomes important to study and evaluate their quantitative and qualitative differences. The operational requirements of the Intelligent River\textsuperscript{\textregistered} middleware system motivated us to compare the choices beyond the standard benchmarks to bring out the unique benefits and limitations of the virtualization approaches.


This thesis evaluates representative implementations of each approach: (i) full-virtualization - \textit{KVM}, (ii) para-virtualization - \textit{Xen}, and (iii) container-based virtualization - \textit{LXC}. First, this thesis describes the scalable and resilient design of the Intelligent River\textsuperscript{\textregistered} middleware system used as a reference application to evaluate the virtualization platforms. Second, this thesis describes the deployment of the application components on each of the test environments. Third, this thesis assesses the benefits and limitations of each based on their virtualization overhead, resource entitlement and isolation facilities, operational flexibility, scalability, and security.


%This thesis evaluates the effectiveness of a representative each of full-virtualization - KVM, para-virtualization - Xen and container-based virtualization - LXC. First, I describe the design of the Intelligent River middleware system that motivated us to compare the choices. Second, I describe the deployment of the middleware system on each of the test environments. Third, I discuss their quantitative and qualitative differences.

The study presented in this thesis provides an improved understanding of available virtualization technologies. The results will be useful to  architects in leveraging the best virtualization platform for a given application.

% based on their virtualization overhead, resource entitlement and isolation facilities, operational flexibility, scalability and security.

%Wireless sensor networks are used to gather meaningful data and to enable important applications. They must satisfy some basic requirements. Specifically, they must be \textit{maintenance free}, \textit{inexpensive}, \textit{reliable}, and \textit{scalable}. The devices used in these networks are deployed in the hundreds to thousands and rely on battery power. It is expensive to change these batteries in such large networks, both in terms of battery cost and personnel time. The cost of an individual device plays an important role in the cost of a sensor network. Further, if these networks are deployed in safety critical contexts, we cannot risk having incorrect data or missing important data. Finally, sensor networks must be able to accommodate new devices and gracefully handle device failures.

%This thesis describes a hardware/software solution for wireless sensor networks which is maintenance free, inexpensive, reliable, and scalable. In this thesis, I present a wireless sensing device which harvests solar energy and stores it in a Li-Ion battery. The device works on solar energy during the daytime and relies on battery power during the night. This addresses the problem of maintaining remote devices. The design is also focused on reducing component costs. The cost is low enough to discard the individual devices without significant concern. Finally, using these devices, I present a network protocol and reference implementation which makes data reception reliable, while supporting network scalability.

% To summarize, this thesis describes a system which is maintenance free, a system which is inexpensive, a system which is reliable, and a system which is scalable.